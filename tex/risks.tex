This section should contain a list of at least 5 of the most critical risks related to your project. Additionally, the probability of occurrence, size of loss, and risk exposure should be listed. For size of loss, express units as the number of days by which the project schedule would be delayed. For risk exposure, multiply the size of loss by the probability of occurrence to obtain the exposure in days. For example:

The following high-level risk census contains identified project risks with the highest exposure. Mitigation strategies will be discussed in future planning sessions.

\begin{table}[h]
\resizebox{\textwidth}{!}{
\begin{tabular}{|l|l|l|l|}
\hline
 \textbf{Risk description} & \textbf{Probability} & \textbf{Loss (days)} & \textbf{Exposure (days)} \\ \hline
 Downtime on the printer causes schedules to slip  & 0.40 & 1 & 0.4 \\ \hline
 Downtime on the Raspberry Pi causes schedules to slip  & 0.30 & 1 & 0.3 \\ \hline
 Hardware errors cause schedules to slip  & 0.20 & 2 & 0.4 \\ \hline
 Running out of filament causes schedules to slip  & 0.40 & 1 & 0.4 \\ \hline
 Power cut or network drop cause schedules to slip & 0.10 & 1 & 0.1 \\ \hline
\end{tabular}}
\caption{Overview of highest exposure project risks} 
\end{table}