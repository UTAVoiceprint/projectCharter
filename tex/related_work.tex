There are a few options for voice-activated printers on the market now.
\begin{itemize}
  \item HP announced a general platform for their printers earlier in 2018 that uses the Google Cloud and Amazon Alexa and enables functionality such as asking "Am I low on blank ink?" Or specifying "Print three copies" \cite{HP2018}.
  \item Xerox AltaLink devices utilize IBM's Watson technology to skip input screens and icons and perform tasks such as scanning, emailing, and initiating service requests through voice commands \cite{Xerox2018}.
\end{itemize}
These solutions are only designed for traditional printers, and are not suitable for printing 3D objects.
\begin{itemize}
  \item Yahoo Japan has developed a voice-activated 3D printer for visually-impaired children \cite{Yahoo2013}.
\end{itemize}
This implementation is only intended for simple shapes and is not suitable for complex 3D printing.
\begin{itemize}
  \item XYZprinting has a feature on it's Da Vinci Color AiO 3D printer to receive certain voice commands \cite{XYZ2018}.
\end{itemize}
This product is more expensive than would be feasible for most end users, and the voice commands are limited to simple instructions.